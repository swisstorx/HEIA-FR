\documentclass[12pt]{article}
  \usepackage[francais]{babel}
  \AddThinSpaceBeforeFootnotes % à insérer si on utilise \usepackage[french]{babel}
  \FrenchFootnotes % à insérer si on utilise \usepackage[french]{babel}
  \usepackage[T1]{fontenc}
  \usepackage[utf8]{inputenc}
  \usepackage{graphicx}
  \usepackage[left=2.5cm,right=2.5cm,top=2.5cm,bottom=2.5cm]{geometry}
  \usepackage{array}
  \usepackage{booktabs}
  \usepackage[squaren,Gray]{SIunits}  % Unité ex: $\unit{5 \cdot 10^{-6}}{\meter}$
  \usepackage{colortbl}               % Pour les couleur des cellules (tableau)
  \usepackage{amsmath}				  % Pour les formules mathématiques
  \usepackage{upgreek}                % Pour les lettres greque
  %\usepackage{fullpage}	          % plus petites marges
  \usepackage{verbatim}				  % Pour de long commentaires
  \usepackage[lofdepth,lotdepth]{subfig}       % Faire des sous-figures
  \usepackage{url}
  \usepackage{colortbl}               % pour les couleur des cellules (tableau)
  \usepackage{indentfirst}
  \usepackage{multirow}
  \usepackage{xfrac}
  \usepackage{wrapfig}
  \usepackage{enumitem}               % Liste personnalisée
  \frenchbsetup{StandardLists=true}   % Empêche conflits entre enumitem et babel
  \usepackage{placeins}   % place une barrière pour que l'image/table soit derrière \FloatBarrier
  \usepackage{lastpage} 
  \usepackage{titling}
  \usepackage{lmodern}
  \usepackage{booktabs}
  \usepackage{etoolbox}
  \usepackage[most]{tcolorbox}
  
  
  %Change la taille de police
  \newcommand\ChangeRT[1]{\noalign{\hrule height #1}}
  
  
  %Création  d'une nouvelle commande pour faire référence à une Figure
  %Exemple : \appelFigure{schema} donne : Figure 1 (en italique)
  \newcommand{\appelFigure}[1]{
    \textit{Figure \ref{#1}}
  }
      
  %%Création commande pour insérer image avec nom de figure directement
  %\newcommand{nomDeTaCommande}[nombreArguments]{CodeLaTeX}
  %\insertImage[position]{image_path}{scale}{Titre_figure}{label}
  \newcommand{\insertImage}[5][center]{
      \begin{#1}
      \includegraphics[scale=#3]{#2}
      \captionof{figure}{#4} 
      \label{#5}
      \end{#1}
  }

  % Affichage des frames pour commande cisco
  \newtcblisting{cisco}[1][]{size=fbox, listing only, listing options={style=tcblatex,basicstyle=\ttfamily\scriptsize,tabsize=2,language=sh},title=#1}

  %En-tête et pied de page personalisé
  \usepackage{fancyhdr}
  \pagestyle{fancy}
  \fancyhf{}
  \setlength\parindent{0pt} %Supprime les alinéa
  \setlength{\parskip}{8pt} %Augmente l'espace entre paragraphe
  %Bottom numbering page
  \renewcommand{\headrulewidth}{1pt}
  \fancyhead[L]{\includegraphics[scale=.2]{Img/heia-fr-logo.png}}
  \fancyhead[R]{\theauthor}
  
  \renewcommand{\footrulewidth}{1pt}
  \fancyfoot[R]{\textbf{Page \thepage\ sur \pageref{LastPage}}} 
  \fancyfoot[L]{\leftmark}

  \setlength\parindent{0pt} %Supprime les alinéa
  \setlength{\parskip}{8pt} %Augmente l'espace entre paragraphe

\title{nom de projet} 
\author{\textsl{Prénom 1} \textsc{Nom 1}\\
\textsl{Prénom 2} \textsc{nom 2}}
\date{}

\begin{document}
    \begin{titlepage}
        \begin{center}
            \includegraphics[scale=.4]{Img/heia-fr-logo.png}\\[1.3cm]
            
            \rule{\linewidth}{0.3mm} \\[0.3cm]
            {\huge \bfseries Branche\\[0.5cm]} 
           % {\Large Effet photoélectrique}\\[0.2cm]
            {\Large  Sujet du cours // labo }
            \rule{\linewidth}{0.3mm} \\[0.8cm]
            \noindent
            \begin{minipage}[t]{0.4\textwidth}
                \begin{flushleft} \large
                    \emph{Auteur(s) :}\\
                    \theauthor
                \end{flushleft}
            \end{minipage}
            \begin{minipage}[t]{0.4\textwidth} 
                \begin{flushright} \large
                    \emph{Professeur:}\\
                    \textsl{Prénom} \textsc{Nom}\\ 
                \end{flushright} 
                \vfill
            \end{minipage}\\[1.3cm]
            \includegraphics[scale=0.6]{Img/1.JPG}\\[1.5cm]
            \vspace*{1\baselineskip}
            \today \\[0.7cm]
        \end{center}
    \end{titlepage}
    \tableofcontents
    \clearpage
% \insertImage{Img/photo.PNG}{0.8}{Schéma explicatif}{EPschemaexpli}
%
%\section{voici comment faire un chapitre}
%
%    \subsection{Voici comment faire un sous chapitre}
%
%\section{des commandes}
%    \subsection{commande cisco}
%    
%    \begin{cisco}[une commande sympa type cisco]
%       un texte lambda
%    \end{cisco}
%    
%    \subsection{commande cisco allégée}
%    
%    \begin{cisco}
%       un texte lambda
%   \end{cisco}
%    
%    
%   \subsection{commande pour mettre une image}
%    
%    % \insertImage{Img/1.PNG}{echelle pour l'image (source = 1)}{texte dessous l'image}{référence vers l'objet
%    \insertImage{Img/1.JPG}{0.6}{voici une image}{myImg}


\section{Conception Header File}

\subsection{Pragma once}

Il permet d’éviter un import multiple de header files en incluant une fois uniquement les
fichiers dans la compilation et peut être accompagné des commandes preprocessor ifndef
symbol define

\subsection{ifndef}

Cette commande sert à éviter les boucles d'appel (si la classe A appelle la B et la B appelle la A)
\subsection{Exemple}
\begin{lstlisting}
    #pragma once
    #ifndef SERPENTINE_H
    #define SERPENTINE_H
    
    extern void serpentine_init();
    
    extern void serpentine_process();
    
    extern void serpentine_reset();
    
    #endif
\end{lstlisting}

%fait%fait%fait%fait%fait

%fait%fait%fait%fait%fait

%fait%fait%fait%fait%fait

%fait%fait%fait%fait%fait
\section{Appel fonction C}








\section{Décrire le passage d’arguments par valeur et par référence lors d’appel de
fonctions en C}

\subsection{passage d'arguments par valeur}

Dans ce cas, on passe une valeur à notre fonction, une copie locale que l'on donne à notre fonction, pour effectuer différents traitements.

\begin{itemize}
    \item Le sous-programme reçoit une copie de la donnée
    \item Il peut la lire et la modifier sans que la donnée originale ne soit altérée
    \item Il existe deux positions mémoires distinctes
    \item La modification de l'une des positions n'a aucune conséquence sur l'autre position.   
\end{itemize}

\textbf{Exemple: } $int fnct1 (int a, char c)$

Mais cette methode a quelques inconvénients, on ne peut pas retourner plusieurs éléments. c'est pourquoi en C on peut procéder au passage par référence.


\subsection{passage d'arguments par référence lors d’appel de fonctions en C}

Pour cette methode on donne l'adresse de notre variable en paramètre de notre fonction. on dit que l'on crée un pointeur qui pointe sur l'adresse de notre variable

\begin{itemize}
    \item Le sous-programme reçoit l'adresse de la donnée
    \item Cette donnée est ainsi partagée entre la fonction appelante et la fonction appelée
    \item La donnée ne se situe que dans une seule position mémoire
    \item La modification du paramètre par la fonction appelée est visible par la fonction appelante
\end{itemize}

\textbf{Exemple: } $int fnct2 (char* s, int* b);$

%fait%fait%fait%fait%fait

%fait%fait%fait%fait%fait

%fait%fait%fait%fait%fait

%fait%fait%fait%fait%fait

\section{Différencier les fonctions globales des fonctions locales dans une application codée
en C}

Les variables globales sont définies à l'extérieur de la fonction. Même si elles sont à l'extérieur de notre fonction elles sont quand même accessibles.

Les variables définies à l'intérieur de notre fonction ne sont pas accessibles à l'extérieur.

Mais cependant, on peut modifier une variables se trouvant à l'extérieur de notre fonction via l'utilisation de pointeurs. Il faut toutefois que le pointeur nous soie donné en paramètre de notre fonction.

\subsection{Tips}

Il est déconseillé de laisser l'accès à ces variables à nos fonctions. celà a pour effet de crée des effets de bords non voulus.


%fait%fait%fait%fait%fait

%fait%fait%fait%fait%fait

%fait%fait%fait%fait%fait

%fait%fait%fait%fait%fait

\section{Différencier les variables globales des variables locales et des variables
rémanentes dans une application codée en C}

wallah c'est pareil que le point précédent.











\section{Manipuler correctement les types complexes (énumérations, tableaux, structures,
unions, …) de C}











\section{Concevoir une interface C permettant d’accéder aux registres d’un périphérique}
    








\section{Manipuler correctement les pointeurs en C}






\section{Allouer et restituer correctement des objets dynamiques en C}
 






\section{décrire et utiliser les conversions des types en C}





\section{Manipuler correctement les pointeurs de fonction en C}







\section{Expliquer les conditions pour qu’une fonction C soit réentrante}

\begin{itemize}
    \item Ne doit pas utiliser de données statiques (globales) non-constantes
    \item Ne doit retourner l’adresse de données statiques (globales) non-constantes
    \item Ne doit traiter que des données fournies par le programme appelant
    \item Ne doit pas modifier son propre code
    \item Ne doit pas appeler des sous-programmes non-réentrants
\end{itemize}


\end{document}