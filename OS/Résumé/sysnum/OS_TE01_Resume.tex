\documentclass[12pt]{article}
  \usepackage[francais]{babel}
  \AddThinSpaceBeforeFootnotes % à insérer si on utilise \usepackage[french]{babel}
  \FrenchFootnotes % à insérer si on utilise \usepackage[french]{babel}
  \usepackage[T1]{fontenc}
  \usepackage[utf8]{inputenc}
  \usepackage{graphicx}
  \usepackage[left=2.5cm,right=2.5cm,top=2.5cm,bottom=2.5cm]{geometry}
  \usepackage{array}
  \usepackage{booktabs}
  \usepackage[squaren,Gray]{SIunits}  % Unité ex: $\unit{5 \cdot 10^{-6}}{\meter}$
  \usepackage{colortbl}               % Pour les couleur des cellules (tableau)
  \usepackage{amsmath}				  % Pour les formules mathématiques
  \usepackage{upgreek}                % Pour les lettres greque
  %\usepackage{fullpage}	          % plus petites marges
  \usepackage{verbatim}				  % Pour de long commentaires
  \usepackage[lofdepth,lotdepth]{subfig}       % Faire des sous-figures
  \usepackage{url}
  \usepackage{colortbl}               % pour les couleur des cellules (tableau)
  \usepackage{indentfirst}
  \usepackage{multirow}
  \usepackage{xfrac}
  \usepackage{wrapfig}
  \usepackage{enumitem}               % Liste personnalisée
  \frenchbsetup{StandardLists=true}   % Empêche conflits entre enumitem et babel
  \usepackage{placeins}   % place une barrière pour que l'image/table soit derrière \FloatBarrier
  \usepackage{lastpage} 
  \usepackage{titling}
  \usepackage{lmodern}
  \usepackage{booktabs}
  \usepackage{etoolbox}
  \usepackage[most]{tcolorbox}
  
  
  %Change la taille de police
  \newcommand\ChangeRT[1]{\noalign{\hrule height #1}}
  
  
  %Création  d'une nouvelle commande pour faire référence à une Figure
  %Exemple : \appelFigure{schema} donne : Figure 1 (en italique)
  \newcommand{\appelFigure}[1]{
    \textit{Figure \ref{#1}}
  }
      
  %%Création commande pour insérer image avec nom de figure directement
  %\newcommand{nomDeTaCommande}[nombreArguments]{CodeLaTeX}
  %\insertImage[position]{image_path}{scale}{Titre_figure}{label}
  \newcommand{\insertImage}[5][center]{
      \begin{#1}
      \includegraphics[scale=#3]{#2}
      \captionof{figure}{#4} 
      \label{#5}
      \end{#1}
  }

  % Affichage des frames pour commande cisco
  \newtcblisting{cisco}[1][]{size=fbox, listing only, listing options={style=tcblatex,basicstyle=\ttfamily\scriptsize,tabsize=2,language=sh},title=#1}

  %En-tête et pied de page personalisé
  \usepackage{fancyhdr}
  \pagestyle{fancy}
  \fancyhf{}
  \setlength\parindent{0pt} %Supprime les alinéa
  \setlength{\parskip}{8pt} %Augmente l'espace entre paragraphe
  %Bottom numbering page
  \renewcommand{\headrulewidth}{1pt}
  \fancyhead[L]{\includegraphics[scale=.2]{Img/heia-fr-logo.png}}
  \fancyhead[R]{\theauthor}
  
  \renewcommand{\footrulewidth}{1pt}
  \fancyfoot[R]{\textbf{Page \thepage\ sur \pageref{LastPage}}} 
  \fancyfoot[L]{\leftmark}

  \setlength\parindent{0pt} %Supprime les alinéa
  \setlength{\parskip}{8pt} %Augmente l'espace entre paragraphe
 
\title{Résumé OS de mort} 
\author{\textsl{Marc} \textsc{Roten}}
\date{}


\begin{document}
    \begin{titlepage}
        \begin{center}
            \includegraphics[scale=.4]{Img/heia-fr-logo.png}\\[1.3cm]
            
            \rule{\linewidth}{0.3mm} \\[0.3cm]
            {\huge \bfseries Operating Systems\\[0.5cm]} 
            {\Large  Résumé TE01 }
            \rule{\linewidth}{0.3mm} \\[0.8cm]
            \noindent{}
            \begin{minipage}[t]{0.4\textwidth}
                \begin{flushleft} \large
                    \emph{Auteur :}\\
                    \theauthor
                \end{flushleft}
            \end{minipage}
            \begin{minipage}[t]{0.4\textwidth} 
                \begin{flushright} \large
                    \emph{Professeur:}\\
                    \textsl{Jacques} \textsc{ Supcik}\\ 
                \end{flushright} 
                \vfill
            \end{minipage}\\[1.3cm]
            \includegraphics[scale=0.4]{Img/title.jpg}\\[1.5cm]
            \vspace*{1\baselineskip}
            \today \\[0.7cm]
        \end{center}
    \end{titlepage}
    \tableofcontents
    \clearpage
% \insertImage{Img/photo.PNG}{0.8}{Schéma explicatif}{}

% \section{voici comment faire un chapitre}

%     \subsection{Voici comment faire un sous chapitre}

% \section{des commandes}
%     \subsection{commande cisco}
    
%     \begin{cisco}[une commande sympa type cisco]
%       un texte lambda
%     \end{cisco}
    
%     \subsection{commande cisco allégée}
    
%     \begin{cisco}
%       un texte lambda
%     \end{cisco}
    
    
%     \subsection{commande pour mettre une image}
    
%     % \insertImage{Img/1.PNG}{echelle pour l'image (source = 1)}{texte dessous l'image}{référence vers l'objet
%     \insertImage{Img/1.JPG}{0.6}{voici une image}{myImg}
%       \insertImage{Img/27.JPG}{0.6}{}{}

\section{THCHAP1 Introduction}%done
Petit résumé incoming.
\section{THCHAP2 Concept de base}%done
    Avec un niveau d'abstraction, on peut considérer le graphique suivant 
    \insertImage{Img/1.JPG}{0.6}{Niveau d'abstraction}{}

    Quand un processus crée un autre processus, on peut le représenter sous forme d'arbre comme suit: 

    \insertImage{Img/2.JPG}{0.6}{Arbre de processus}{}

    \subsection{Les appels systemes}

    \insertImage{Img/3.JPG}{0.6}{appels systèmes simplifiés}{}

    \subsubsection{Gestion des processus}
    \insertImage{Img/4.JPG}{0.6}{les différenta appels systemes}{}

    \subsubsection{Gestion de fichier}
    \insertImage{Img/5.JPG}{0.6}{les différenta appels systemes}{}

    \subsubsection{Gestion répertoires}

    \insertImage{Img/6.JPG}{0.6}{les différenta appels systemes}{}
\subsection{Structure d'un OS}
    \subsubsection{Système monolithiques}
        Systèmes tels que linux
    \subsubsection{MicroNoyaux}
    \insertImage{Img/7.JPG}{0.6}{MicroNoyaux}{}
    \subsubsection{Modèle client serveur}
    \insertImage{Img/8.JPG}{0.6}{client-serveur}{}
    \subsubsection{Machines virtuelles}
    \insertImage{Img/9.JPG}{0.6}{VM}{}




\newpage
\section{THCHAP3 Processus et threads}%done

Les processus sont:

\begin{itemize}
    \item le concept central d'un OS
    \item Abstraction d'un programme en cours d'execution
    \item Concurrence même avec un seul processeur
\end{itemize}

\subsection{Définitions simplifiées}
\begin{itemize}
    \item \textbf{Le programme} est la recette de cuisine (algorithme)
    \item \textbf{Le processeur} est le cuisinier qui suit la recette
    \item \textbf{le processus} est l’action de cuisiner (activité)
    \item \textbf{Si le cuisiner se brûle, il remplace le livre de cuisine par celui des premiers secours et se soigne}
\end{itemize}

\subsection{Naissance d'un processus}

Les evénements qui induisent la création de processus sont:
\begin{itemize}
    \item initialisation du systeme
    \item Execution d'un appel système de création de processus en cours d'execution (Fork)
    \item Requête utilisateur sollicitant éa création de nouveaux processus.
    \item Lancement d'un travail en traitement par lots (batch)
\end{itemize}

\subsection{Mort d'un processus}
\begin{itemize}
    \item Arrêt normal, volontaire
    \item arrêt pour erreur, volontaire
    \item Fatal error, involontaire
    \item autre process qui arrête notre processus
\end{itemize}

\subsection{Machine d'état processus}
\insertImage{Img/10.JPG}{0.6}{FSM Processus}{}
Le Scheduler s'occuppe de dire quel process travaille et quand le process travaille. C'est en quelque sorte le chef d'orchestre dans tout ce bordel.

\subsection{Utilisation Processus / process en mémoire}
\insertImage{Img/11.JPG}{0.6}{FSM Processus}{}
\subsection{Les threads}

\subsubsection{Schéma conceptuel Threads}
\insertImage{Img/12.JPG}{0.6}{Concept threads}{}

\subsubsection{Code Threads}
\insertImage{Img/13.JPG}{0.6}{Exemple de threads}{}

\subsubsection{Manière de créer un serveur }
\insertImage{Img/14.JPG}{0.6}{How to build a server and get a life}{}

\subsubsection{MultiThread}
\insertImage{Img/15.JPG}{0.6}{le fameux multi srède}{}

\subsection{Conditions pour eviter les conditions de concurrence}
\begin{itemize}
    \item deux process ne doivent pas se trouver dans leurs sections critiques.
    \item Il ne daut pas faire de supposition quant à la vitesse ou au nombre de processeur mis en oeuvre
    \item Aucun processus s'executant à l'intérieur de la section critique ne doit bloquer d'autres processus.
    \item Aucun processeur ne doit attendre indéfiniment pour pouvoir entrer dans sa section critique.
\end{itemize}
\insertImage{Img/16.JPG}{0.6}{La Zone critique, et si on connait pas, on trikite pas}{}















\newpage
\section{THCHAP4 L'ordonnancement des processus}

C'est quand qu'on ordonnance Jamie? 

Et bien c'est très simple, on le fait: 

\begin{itemize}
    \item Lors de la création d'un process, parent ou enfant qui prend le contrôle?
    \item Lorsqu'un process se termine
    \item Lorsqu'un process se met en attente d'un event sur une I/O 
    \item Lors d'une interruption faite par un periphérique
    \item Lors d'une interruption par l'horloge pour des processus préemptif
\end{itemize}

\subsection{Système d'exploitation préemptif}
C'est la capacité qu'a un systême d'exploitation multiTache à executer ou stopper des tâches planifiées en cours.
\subsection{Catégories d'ordonnancement d'algorithmes}

\begin{itemize}
    \item \textbf{Non préemptifs: }traitement par lot (batch) non-préemptif, ou alors avec une très grande période.
    \item \textbf{Interactifs: }interactifs. préemptif avec courte période
    \item \textbf{Collaboratifs: }temps réel. Préemption non nécessaires. les processus sont connus et collaborent. 
\end{itemize}

\subsection{Objectifs: ordonnancement des process}
\begin{itemize}
    \item \textbf{Equité: }attribuer à chaque processus un temps processeur équitable
    \item \textbf{Application de la politique: }faire en sorte que la politique soit bien appliquée, \textbf{MAKE SCHEDULER GREAT AGAIN}
    \item \textbf{Equilibre: }faire en sorte que toutes les parties du systême soient occupées.
\end{itemize}

\subsubsection{traitement par lots (batch)}
\begin{itemize}
    \item \textbf{Capacité de traitement: }Optimiser le nombre de jobs à l'heure
    \item \textbf{Délai de rotation: }réduire le délai entre soumission et achèvement
    \item \textbf{Utilisation du CPU: }Faire en sorte que le processeur soit occupé en permanence.
\end{itemize}

\subsubsection{systêmes interactifs}
\begin{itemize}
    \item \textbf{Temps de réponse: }répondre rapidement aux requêtes
    \item \textbf{Proportionnalité: }satisfaire aux attentes des utilisateurs
\end{itemize}

\subsubsection{Systême en temps réel}
\begin{itemize}
    \item \textbf{Respecter les délais: }Eviter de perdre des données
    \item \textbf{Prévisibilité: }Eviter la dégradation de la qualités dans les systémes multimédias
\end{itemize}
\subsubsection{Mashup ALL}
    \insertImage{Img/17.JPG}{0.8}{Mashup all}{}
\subsection{Traitement par lots}
\begin{itemize}
    \item FIFO, premier arrivé, premier servi \textbf{NO PREEMPTION}
    \item Shortest job first
    \item Shortest remaining time next
\end{itemize}

\subsubsection{Pourquoi les plus courts en premiers}
Dans ce cas ci
\begin{itemize}
    \item on admet que le temps de traitement des processus est connu à l'avance par l'ordonnanceur.
    \item Le but est d'optimiser le délai de rotation so tous les jobs sont soumis en même temps
\end{itemize}

\subsection{Les systêmes interactifs}
\begin{itemize}
    \item L'ordonnancement est de type \textbf{Round Robin Scheduling}
    \item Ordonnancement par priorités \textbf{Priority Scheduling}
    \item Files d'attentes multiples
    \item Processus le plus court apres\textbf{Shortest process next}
    \item ordonnancement garanti
    \item ordonnancement par tirage au sorte
    \item ordonnancement équitable
\end{itemize}

\subsection{Round Robin}
\insertImage{Img/18.JPG}{0.8}{Round robin}{}
\insertImage{Img/19.JPG}{0.8}{Round robin,explication}{}

\subsection{Ordonnancement par priorité}
\insertImage{Img/20.JPG}{0.8}{Par priorité}{}

\subsection{File d'attentes multiples}
\insertImage{Img/21.JPG}{0.8}{files d'attentes multiples}{}

\subsection{Shortest process next}
\insertImage{Img/22.JPG}{0.8}{Shortest process next}{}

\subsection{Ordonnement Garanti}
\insertImage{Img/23.JPG}{0.8}{Shortest process next}{}

\subsection{Par tirage au sort}
\insertImage{Img/24.JPG}{0.8}{Shortest process next}{}

\subsection{Ordonnancement équitable}
\insertImage{Img/25.JPG}{0.8}{Shortest process next}{}













\newpage
\section{THCHAP6 Gestion de la mémoire}
Solution pour les programmes plus gros que la taille de la mémoire
\subsection{Le Swapping}
L’allocation mémoire change au gré des processus qui viennent en mémoire et qui la quittent. Les zones grisées indiquent que la mémoire est inutilisée. Comme illustré ci dessous.

\insertImage{Img/26.JPG}{0.6}{Swapping}{}
\subsubsection{Problème induits}
\insertImage{Img/27.JPG}{0.6}{problèmes}{}

\newpage
\subsection{Bitmap}
\insertImage{Img/28.JPG}{0.6}{Bitmap}{}
\insertImage{Img/29.JPG}{0.6}{Bitmap specs}{}

\subsection{Linked List} 

\insertImage{Img/30.JPG}{0.6}{Linked List specs}{}

Avec les listes chaînées  on peut implémenter différents algorithmes:
\begin{itemize}
    \item \textbf{First Fit: }on itère depuis le début de la liste et on prend le premier qui passe
    \item \textbf{Next Fit: } On itère depuis l'endroit pointé actuellement et on prend le premier qui vient
    \item \textbf{Best Fit: } On itère dans toute la liste et on prend le meilleur
    \item \textbf{Worst Fit: }On recherche le bloc le plus grand parmis toute la liste
\end{itemize}

\subsection{Best and WorstFit}
Si on doit utiliser ces types d'algorithmes, il est judicieux de trier la liste par taille de blocs


\subsection{Concept de mémoire virtuelle}
\begin{itemize}
    \item La mémoire virtuelle permet la multiprogrammation.
    \item La memoire virtuelle permet d'avoir \textbf{Plus de mémoire physique que virtuelle}
    \item L'espace d'adressage est découpé en pages (suites d'addresses contigues)
    \item chaque programme a son espace d'adressage (de 0 à n)
\end{itemize}
\subsection{La pagination}
C'est la MMU (memory management unit) qui s'en charge. 
\insertImage{Img/31.JPG}{0.6}{MMU}{}
Ci-dessous Exemple trivial de mapping entre physique et virtuel.
\insertImage{Img/32.JPG}{0.6}{Mapping virtuel <---> physique}{}
\begin{itemize}
    \item Les pages de la mémoire virtuelle sont mappées sur la mémoire physique.
    \item Il n'est pas nécessaire d'avoir toutes les pages mappées pour executer un processus.
    \item La multiprogrammation profite des I/O générées par le Swapping pour attribuer les ressources CPU à d'autres processus.
\end{itemize}

\subsubsection{CPU essaie acceder à page virtuelle occupée}
\insertImage{Img/33.JPG}{0.6}{}{}

\subsection{Speedup the pagination}
Tout système de pagination doit considérer ces deux éléments :
\begin{itemize}
    \item La correspondance de l’adresse virtuelle vers l’adresse physique doit être rapide, car elle est utilisée à chaque accès mémoire (donc très souvent).
    \item Si l’espace d’adressage est grand, la table des pages sera grande.Pour un espace d’adressage sur 32 bit et des pages de 4 kiByte, il y a 1M de pages.
    \item De plus, chaque processus à besoin de sa propre table de pages !
\end{itemize}

\subsection{La TLB, translation lookaside buffer}
Les références dans la table sont groupées et seule une petite partie de la table est réelement utilisée.

\insertImage{Img/34.JPG}{0.9}{With an empty TLB}{}
\insertImage{Img/35.JPG}{0.5}{TLB Hit}{}
\insertImage{Img/36.JPG}{0.6}{TLB Miss}{}

\section{Conclusion}
Si vous avez aimé mon résumé, faites un git clone de mon Git. Suivez moi sur gitlab.forge.heia-fr.ch github and iLoveFreeSoftware.com.


\end{document}