\documentclass[12pt, openany]{article}
\usepackage[utf8]{inputenc}
\usepackage[T1]{fontenc}
\usepackage[a4paper,left=2cm,right=2cm,top=2cm,bottom=2cm]{geometry}
\usepackage[frenchb]{babel}
\usepackage{libertine}
\usepackage[pdftex]{graphicx}

\setlength{\parindent}{0cm}
\setlength{\parskip}{1ex plus 0.5ex minus 0.2ex}
\newcommand{\hsp}{\hspace{20pt}}
\newcommand{\HRule}{\rule{\linewidth}{0.5mm}}

\begin{document}

\begin{titlepage}
  \begin{sffamily}
  \begin{center}

    % Upper part of the page. The '~' is needed because \\
    % only works if a paragraph has started.
    \includegraphics[scale=1]{logo_school.PNG}~\\[1.5cm]
    %branche
    \textsc{\LARGE Réseau IP}\\[2cm]
    %classe
    \textsc{\Large T2A}\\[1cm]
    %orientation
    \textsc{\Large Réseau et Sécutité}\\[1cm]
  
    \HRule \\[0.4cm] 
    % Title
    { \huge \bfseries TP4 introduction aux services TCP/IP:DNS\\[0.4cm] }
    \HRule \\[4cm]
    %nom prenom
    \textsc{\Large Josué Tille}\\[1cm]
    \textsc{\Large Roten Marc}\\[1cm]
    \vfill

    % Bottom of the page
    {\large  2017/2018}

  \end{center}
  \end{sffamily}
\end{titlepage}
\tableofcontents
\newpage
\section{Introduction}

Ce travail pratique a pour but de nous introduire aux services TCP/IP, plus particulièrement le protocole DNS, qui est omniprésent, dans notre vie de tous les jours. Effectivement c'est plus facile de se souvenir d'un nom de domaine comme tlabs.tic.heia-fr.ch plutôt que 160.98.31.32.

\section{Mise en place}
\subsection{Configuration du NoteBook}
P1	Documenter et valider le bon fonctionnement de votre maquette. Pour cela, utiliser les commandes ‘nslookup’2 et ‘ping’ sur votre notebook.

P2	Décrivez les différents paramètres de configurations utilisés dans db.pcltexx.ch.zone et dans db.30.98.160.in-addr.zone 

P3	Quels sont les paramètres qu’il faut configurer au minimun lorsque vous voulez gérer et configurer un domaine



P4	Quels est l’organisme qui gère les domaines .ch ? Comment obtenez-vous cette information ?

P5	A quoi sert le fichier /etc/bind/db.root ?

\section{Analyse des protocoles observés}

\subsection{Analyse des échanges DNS}

P6	Quels sont les protocoles de couche 2, 3 et 4 utilisés pour l’échange DNS ? Indiquez le champ dans chacune des couches qui vous permet de définir le protocole qui est transporté.

P7	Quels sont les interlocuteurs de votre notebook et de la machine Linux pour les dialogues DNS ? Quelles sont leurs adresses IP ? Combien de trames provenant et à destination de votre notebook avez-vous enregistrées ? Commentez !

P8	Quels sont les types de message DNS observés ?  

P9	Dessinez les échanges observés entre le client (votre notebook), le serveur DNS et Internet en fonction du temps (diagramme en flèche), commentez ! 

P10	Où se trouve l’information demandée ? Quelles sont les réponses du serveur DNS ? 

\subsection{Requêtes inverses}

P11	Quels sont les types de message DNS observés ?  

P12	Où se trouve l’information demandée ? Quelles sont les réponses du serveur DNS ?
\subsection{Requêtes itératives}

P13	Dessiner le diagramme en flèches des échanges observés. 

P14	Combien de requêtes effectue votre serveur DNS pour résoudre la requête ci-dessus ?

\section{conclusion}

\bibliographystyle{plain}
\bibliography{references}

\end{document}