\documentclass[12pt]{article}
  \usepackage[francais]{babel}
  \AddThinSpaceBeforeFootnotes % à insérer si on utilise \usepackage[french]{babel}
  \FrenchFootnotes % à insérer si on utilise \usepackage[french]{babel}
  \usepackage[T1]{fontenc}
  \usepackage[utf8]{inputenc}
  \usepackage{graphicx}
  \usepackage[left=2.5cm,right=2.5cm,top=2.5cm,bottom=2.5cm]{geometry}
  \usepackage{array}
  \usepackage{booktabs}
  \usepackage[squaren,Gray]{SIunits}  % Unité ex: $\unit{5 \cdot 10^{-6}}{\meter}$
  \usepackage{colortbl}               % Pour les couleur des cellules (tableau)
  \usepackage{amsmath}				  % Pour les formules mathématiques
  \usepackage{upgreek}                % Pour les lettres greque
  %\usepackage{fullpage}	          % plus petites marges
  \usepackage{verbatim}				  % Pour de long commentaires
  \usepackage[lofdepth,lotdepth]{subfig}       % Faire des sous-figures
  \usepackage{url}
  \usepackage{colortbl}               % pour les couleur des cellules (tableau)
  \usepackage{indentfirst}
  \usepackage{multirow}
  \usepackage{xfrac}
  \usepackage{wrapfig}
  \usepackage{enumitem}               % Liste personnalisée
  \frenchbsetup{StandardLists=true}   % Empêche conflits entre enumitem et babel
  \usepackage{placeins}   % place une barrière pour que l'image/table soit derrière \FloatBarrier
  \usepackage{lastpage} 
  \usepackage{titling}
  \usepackage{lmodern}
  \usepackage{booktabs}
  \usepackage{etoolbox}
  \usepackage[most]{tcolorbox}
  
  
  %Change la taille de police
  \newcommand\ChangeRT[1]{\noalign{\hrule height #1}}
  
  
  %Création  d'une nouvelle commande pour faire référence à une Figure
  %Exemple : \appelFigure{schema} donne : Figure 1 (en italique)
  \newcommand{\appelFigure}[1]{
    \textit{Figure \ref{#1}}
  }
      
  %%Création commande pour insérer image avec nom de figure directement
  %\newcommand{nomDeTaCommande}[nombreArguments]{CodeLaTeX}
  %\insertImage[position]{image_path}{scale}{Titre_figure}{label}
  \newcommand{\insertImage}[5][center]{
      \begin{#1}
      \includegraphics[scale=#3]{#2}
      \captionof{figure}{#4} 
      \label{#5}
      \end{#1}
  }

  % Affichage des frames pour commande cisco
  \newtcblisting{cisco}[1][]{size=fbox, listing only, listing options={style=tcblatex,basicstyle=\ttfamily\scriptsize,tabsize=2,language=sh},title=#1}

  %En-tête et pied de page personalisé
  \usepackage{fancyhdr}
  \pagestyle{fancy}
  \fancyhf{}
  \setlength\parindent{0pt} %Supprime les alinéa
  \setlength{\parskip}{8pt} %Augmente l'espace entre paragraphe
  %Bottom numbering page
  \renewcommand{\headrulewidth}{1pt}
  \fancyhead[L]{\includegraphics[scale=.2]{Img/heia-fr-logo.png}}
  \fancyhead[R]{\theauthor}
  
  \renewcommand{\footrulewidth}{1pt}
  \fancyfoot[R]{\textbf{Page \thepage\ sur \pageref{LastPage}}} 
  \fancyfoot[L]{\leftmark}

  \setlength\parindent{0pt} %Supprime les alinéa
  \setlength{\parskip}{8pt} %Augmente l'espace entre paragraphe
\title{Résumé système numérique} 
\author{\textsl{Marc} \textsc{Roten}}
\date{}

\begin{document}
    \begin{titlepage}
        \begin{center}
            \includegraphics[scale=.4]{Img/heia-fr-logo.png}\\[1.3cm]
            
            \rule{\linewidth}{0.3mm} \\[0.3cm]
            {\huge \bfseries Système numérique\\[0.5cm]} 
           % {\Large Effet photoélectrique}\\[0.2cm]
            {\Large  Résumé Chapitre 3 Fonctions combinatoires bascules }
            \rule{\linewidth}{0.3mm} \\[0.8cm]
            \noindent{}
            \begin{minipage}[t]{0.4\textwidth}
                \begin{flushleft} \large
                    \emph{Auteurs :}\\
                    \theauthor
                \end{flushleft}
            \end{minipage}
            \begin{minipage}[t]{0.4\textwidth} 
                \begin{flushright} \large
                    \emph{Professeur:}\\
                    \textsl{Nicolas} \textsc{ Schroeter}\\ 
                \end{flushright} 
                \vfill
            \end{minipage}\\[1.3cm]
            \includegraphics[scale=0.7]{Img/title.jpg}\\[1.5cm]
            \vspace*{1\baselineskip}
            \today \\[0.7cm]
        \end{center}
    \end{titlepage}
    \tableofcontents
    \clearpage
% \insertImage{Img/photo.PNG}{0.8}{Schéma explicatif}{}

% \section{voici comment faire un chapitre}

%     \subsection{Voici comment faire un sous chapitre}

% \section{des commandes}
%     \subsection{commande cisco}
    
%     \begin{cisco}[une commande sympa type cisco]
%       un texte lambda
%     \end{cisco}
    
%     \subsection{commande cisco allégée}
    
%     \begin{cisco}
%       un texte lambda
%     \end{cisco}
    
    
%     \subsection{commande pour mettre une image}
    
%     % \insertImage{Img/1.PNG}{echelle pour l'image (source = 1)}{texte dessous l'image}{référence vers l'objet
%     \insertImage{Img/1.JPG}{0.6}{voici une image}{myImg}

\section{Fonction combinatoire}

Il existe différents types de fonctions combinatoires:
\begin{itemize}
    \item Equation logique
    \item table de vérité
    \item multiplexage
    \item comparaison
    \item encodage de priorité
    \item addition/soustraction
\end{itemize}

\section{Fonctions d'affectation}
\insertImage{Img/1.JPG}{0.6}{différentes fonctions d'affectation}{myImg}
\insertImage{Img/2.JPG}{0.6}{différentes fonctions d'affectation}{myImg}

\subsection{Exemple}
\insertImage{Img/3.JPG}{0.6}{Petit Exemple}{myImg}




\newpage
\section{Décodeur 3 a 8}

\insertImage{Img/4.JPG}{0.5}{Décodeur 3 vers 8}{myImg}

\insertImage{Img/rtl.JPG}{0.5}{Schéma rtl décodeur 3 à 8}{myImg}

\insertImage{Img/code.JPG}{0.5}{Code décodeur 3 à 8}{myImg}

\section{Multiplexeur 4  à 1}
\insertImage{Img/muxrtl.JPG}{0.6}{Schéma rtl Multiplexeur 4 vers 1}{myImg}

\insertImage{Img/muxcode.JPG}{0.6}{Code Multiplexeur 4 vers 1}{myImg}

\section{Comparateur 4 bits}
\insertImage{Img/comprtl.JPG}{0.6}{Schéma RTL Comparateur}{myImg}
\insertImage{Img/compcode.JPG}{0.6}{Code Comparateur}{myImg}

\section{Encodeur de priorité}
le But de l'encodeur de priorité est de déterminer l’indice de l’entrée active ayant le degré de priorité le plus élevé

\subsection{Schéma et TV Encodeur}
\insertImage{Img/5.JPG}{0.6}{schéma encodeur priorité}{myImg}
\insertImage{Img/6.JPG}{0.6}{code encodeur priorité}{myImg}
\insertImage{Img/vueRTL.JPG}{0.5}{Schéma RTL encodeur priorité}{myImg}

\newpage

\section{Les Bascules}
\subsection{Etat initial}
\insertImage{Img/10.JPG}{0.6}{Etat initial bascules}{myImg}
\subsection{Flip Flop}
\insertImage{Img/7.JPG}{0.6}{Description simple DFF}{myImg}
\insertImage{Img/8.JPG}{0.6}{particulatirés de cette definition}{myImg}

\subsection{FlipFlop T}
\insertImage{Img/ex1.JPG}{0.45}{FlipFlop}{myImg}

\subsection{Flip Flop Enable}
\insertImage{Img/9.JPG}{0.6}{Flip Flop Enable}{myImg}

\subsection{DFFb}

\insertImage{Img/ex2.JPG}{0.45}{Flip Flop b}{myImg}

\subsection{D Latch}

\insertImage{Img/ex3.JPG}{0.45}{D-Latch}{myImg}


\section{Reset synchrone vs asynchrone}
\insertImage{Img/11.JPG}{0.6}{Différences entre les deux}{myImg}
\textbf{IL EST NECESSAIRE DE N'UTILISER QUUN SEUL TYPE DE RESET PAR PROJET!!!}

Fonctions d'un Reset
\begin{itemize}
    \item initialiser FSM
    \item initialiser Compteur
    \item initialiser bascules
\end{itemize}

Problèmes liés aux resets asynchrone:
\begin{itemize}
    \item Si plusieurs entrées changent en meme temps, avec un système à reset asynchrone, notre machine d'état ne sera plus déterministe.
\end{itemize}

On peut / doit donc rendre le signal de reset synchrone, pour se faire, procéder comme suit:

\insertImage{Img/12.JPG}{0.6}{asynchrone to synchrone}{myImg}

\section{Bonne partique bien Schroeterproof}

\insertImage{Img/13.JPG}{0.6}{Bonnes pratiques}{myImg}









\end{document}
