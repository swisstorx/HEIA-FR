\documentclass[12pt]{article}
  \usepackage[francais]{babel}
  \AddThinSpaceBeforeFootnotes % à insérer si on utilise \usepackage[french]{babel}
  \FrenchFootnotes % à insérer si on utilise \usepackage[french]{babel}
  \usepackage[T1]{fontenc}
  \usepackage[utf8]{inputenc}
  \usepackage{graphicx}
  \usepackage[left=2.5cm,right=2.5cm,top=2.5cm,bottom=2.5cm]{geometry}
  \usepackage{array}
  \usepackage{booktabs}
  \usepackage[squaren,Gray]{SIunits}  % Unité ex: $\unit{5 \cdot 10^{-6}}{\meter}$
  \usepackage{colortbl}               % Pour les couleur des cellules (tableau)
  \usepackage{amsmath}				  % Pour les formules mathématiques
  \usepackage{upgreek}                % Pour les lettres greque
  %\usepackage{fullpage}	          % plus petites marges
  \usepackage{verbatim}				  % Pour de long commentaires
  \usepackage[lofdepth,lotdepth]{subfig}       % Faire des sous-figures
  \usepackage{url}
  \usepackage{colortbl}               % pour les couleur des cellules (tableau)
  \usepackage{indentfirst}
  \usepackage{multirow}
  \usepackage{xfrac}
  \usepackage{wrapfig}
  \usepackage{enumitem}               % Liste personnalisée
  \frenchbsetup{StandardLists=true}   % Empêche conflits entre enumitem et babel
  \usepackage{placeins}   % place une barrière pour que l'image/table soit derrière \FloatBarrier
  \usepackage{lastpage} 
  \usepackage{titling}
  \usepackage{lmodern}
  \usepackage{booktabs}
  \usepackage{etoolbox}
  \usepackage[most]{tcolorbox}
  
  
  %Change la taille de police
  \newcommand\ChangeRT[1]{\noalign{\hrule height #1}}
  
  
  %Création  d'une nouvelle commande pour faire référence à une Figure
  %Exemple : \appelFigure{schema} donne : Figure 1 (en italique)
  \newcommand{\appelFigure}[1]{
    \textit{Figure \ref{#1}}
  }
      
  %%Création commande pour insérer image avec nom de figure directement
  %\newcommand{nomDeTaCommande}[nombreArguments]{CodeLaTeX}
  %\insertImage[position]{image_path}{scale}{Titre_figure}{label}
  \newcommand{\insertImage}[5][center]{
      \begin{#1}
      \includegraphics[scale=#3]{#2}
      \captionof{figure}{#4} 
      \label{#5}
      \end{#1}
  }

  % Affichage des frames pour commande cisco
  \newtcblisting{cisco}[1][]{size=fbox, listing only, listing options={style=tcblatex,basicstyle=\ttfamily\scriptsize,tabsize=2,language=sh},title=#1}

  %En-tête et pied de page personalisé
  \usepackage{fancyhdr}
  \pagestyle{fancy}
  \fancyhf{}
  \setlength\parindent{0pt} %Supprime les alinéa
  \setlength{\parskip}{8pt} %Augmente l'espace entre paragraphe
  %Bottom numbering page
  \renewcommand{\headrulewidth}{1pt}
  \fancyhead[L]{\includegraphics[scale=.2]{Img/heia-fr-logo.png}}
  \fancyhead[R]{\theauthor}
  
  \renewcommand{\footrulewidth}{1pt}
  \fancyfoot[R]{\textbf{Page \thepage\ sur \pageref{LastPage}}} 
  \fancyfoot[L]{\leftmark}

  \setlength\parindent{0pt} %Supprime les alinéa
  \setlength{\parskip}{8pt} %Augmente l'espace entre paragraphe

\title{Résumé système numérique} 
\author{\textsl{Marc} \textsc{Roten}}
\date{}

\begin{document}
    \begin{titlepage}
        \begin{center}
            \includegraphics[scale=.4]{Img/heia-fr-logo.png}\\[1.3cm]
            
            \rule{\linewidth}{0.3mm} \\[0.3cm]
            {\huge \bfseries Système numérique\\[0.5cm]} 
           % {\Large Effet photoélectrique}\\[0.2cm]
            {\Large  Résumé }
            \rule{\linewidth}{0.3mm} \\[0.8cm]
            \noindent
            \begin{minipage}[t]{0.4\textwidth}
                \begin{flushleft} \large
                    \emph{Auteurs :}\\
                    \theauthor
                \end{flushleft}
            \end{minipage}
            \begin{minipage}[t]{0.4\textwidth} 
                \begin{flushright} \large
                    \emph{Professeur:}\\
                    \textsl{Nicolas} \textsc{ Schroeter}\\ 
                \end{flushright} 
                \vfill
            \end{minipage}\\[1.3cm]
            \includegraphics[scale=0.7]{Img/title.jpg}\\[1.5cm]
            \vspace*{1\baselineskip}
            \today \\[0.7cm]
        \end{center}
    \end{titlepage}
    \tableofcontents
    \clearpage
% \insertImage{Img/photo.PNG}{0.8}{Schéma explicatif}{}

% \section{voici comment faire un chapitre}

%     \subsection{Voici comment faire un sous chapitre}

% \section{des commandes}
%     \subsection{commande cisco}
    
%     \begin{cisco}[une commande sympa type cisco]
%       un texte lambda
%     \end{cisco}
    
%     \subsection{commande cisco allégée}
    
%     \begin{cisco}
%       un texte lambda
%     \end{cisco}
    
    
%     \subsection{commande pour mettre une image}
    
%     % \insertImage{Img/1.PNG}{echelle pour l'image (source = 1)}{texte dessous l'image}{référence vers l'objet
%     \insertImage{Img/1.JPG}{0.6}{voici une image}{myImg}

\section{Chapter 1 : Introduction au VHDL}

\subsection{Procédure}
\insertImage{Img/0.JPG}{0.6}{procédure}{}

\subsection{Signaux types et opérateurs}
\insertImage{Img/1.JPG}{0.8}{Rappel des différents états, types et opérteurs}{}

\subsection{Std logic vector}

\insertImage{Img/2.JPG}{0.8}{STD LOGIC VECTOR}{}

Il est possible de ne pas définir la taille d'un vecteur. Nous
parlons alors de vecteur !!!!! non-contraint !!!!!. Il s'agit d'une application
particulière pour les descriptions réutilisables.

\subsubsection{Les non-contraint}

On peut réaliser des vecteur non-contraint, pour la réutilisabilité du code, mais lors de l'instanciation, il faut rajouter les dites-contraintes. Dans un FPGA, en VHDL, TOUT EST DETERMINISTE.

\subsection{Ordre des opérateurs}
\insertImage{Img/4.JPG}{0.6}{ordre de précédence}{}
\newpage
Exemples ci-dessous:

$$ A <="10111"$$
$$ A ~ ssl ~  1 ->"01110"$$
$$ A  ~ sra ~  2 -> "11101"$$
$$ A  ~ rol ~  3 -> "11101"$$
 
 \subsection{Rappel structure du code}   
   \insertImage{Img/5.JPG}{0.6}{Structure du code}{}
   
 \subsection{Description d'un composant, Entity}
   \insertImage{Img/6.JPG}{0.6}{Description d'un conposant}{}
\subsection{Rappel le la syntaxe pour l'architecture}

\insertImage{Img/7.JPG}{0.6}{architecture}{1}

\subsubsection{Zone de Déclaration}
\insertImage{Img/8.JPG}{0.6}{zone de déclaration}{1}

\subsubsection{Zone de code}
\insertImage{Img/9.JPG}{0.6}{Zone de code}{1}


\newpage
\subsection{Conception avec le VHDL}

$$IL~FAUT~PENSER~CIRCUIT$$
$$Faire~des~descriptions~SIMPLES~et~LISIBLE$$

=====================================

=====================================

Rajouter du texte de la page 23

=====================================

=====================================

done jusqu'à la page 23

%%%%%%%%%%%%%%%%%%%%%%%%%%%%%%%%%%%%%%%%%%%%%%%%%%%%%%%%%%%%%%%%%%%%%%%%%%%%%%%%%%%%%%%%%%%
\subsection{Portabilité}

\insertImage{Img/10.JPG}{0.6}{Zone de code}{1}

\subsection{Outils d'instructions concurrentes.}

\insertImage{Img/11.JPG}{0.6}{Zone de code}{1}

\subsubsection{Affectation}

\insertImage{Img/12.JPG}{0.6}{Zone de code}{1}


En simulation, on peut affecter le meme signal à plusieurs circuit. La synthèse nous laisse uniquement faire 1 Signal sur 1 circuit

\subsubsection{Affectation avec condition}
L'avantage de cette méthode est qu'elle ne crée aucun latch, et en VHDL, il ne faut aucun latch.

\insertImage{Img/13.JPG}{0.6}{Zone de code}{1}

\subsubsection{Affectation de séléction}

\insertImage{Img/14.JPG}{0.6}{Zone de code}{1}

\subsection{Instanciation d'un composant}
\subsubsection{Methode Schroeter}
\insertImage{Img/15.JPG}{0.6}{Zone de code}{1}
\insertImage{Img/16.JPG}{0.6}{Lien avec une bibliothèque}{1}

\subsubsection{Méthode Etudiand}

\insertImage{Img/17.JPG}{0.6}{Lien avec une bibliothèque}{1}


\subsection{Process}

\insertImage{Img/18.JPG}{0.6}{Lien avec une bibliothèque}{1}
\textbf{Un seul bit d un std logic vector bouge, et on réveille toute la liste de sensibilité}

$$se rendort lorsque toutes les instructions
séquentielles (avant le end process / wait)
ont été évaluées.$$

\subsubsection{Déclaration à l'intérieur}

$$interdit d'avoir une variable sans valeur, leur donner une valeur initiale$$

\insertImage{Img/19.JPG}{0.6}{Lien avec une bibliothèque}{1}


\subsection{Instructions Séquentielles}

\insertImage{Img/20.JPG}{0.6}{}{1}


\subsection{Types supplémentaires et conversion}
\textbf{le languqge est fortement typé}

\insertImage{Img/21.JPG}{0.6}{}{1}


\subsection{Bascule D}

\insertImage{Img/22.JPG}{0.6}{}{1}

\subsection{Final State Machine VHDL Model}
\subsubsection{Architecture}
\insertImage{Img/23.JPG}{0.6}{}{1}
\subsubsection{Registre}
\insertImage{Img/24.JPG}{0.6}{}{1}

\subsubsection{Circuit de sortie}
\insertImage{Img/25.JPG}{0.6}{}{1}

\subsubsection{Mémoire}
\insertImage{Img/26.JPG}{0.6}{}{1}
Data est toujours une constantge de type signal. On fait des opérations à l'intérieur des process sur des unsigned.

$$Pas~ besoin~ de~ else$$
$$la ~valeur~ du vsignal~ Datamem~ est~ implicitement~ conservée~ si~ le~ signal~ n'est ~pas~ affecte$$




\newpage
\subsection{IMPORTANT}
\subsubsection{Signal}
\textbf{Un signal c'est un fil, AUCUNE MEMOIRE}

\subsubsection{Un combinatoire}
\textbf{SI LA SORTIE CHANGE IMMEDIATEMENT}

affecter un signal combinatoire à un circuit combinatoire par l'opérateur "<=" lien définitif \textbf{comme si on soudait le signal à un composant}

\textbf{Si un élément de la liste de sensibilité change, INSTANTANEMENT la sortie change}

\subsubsection{Mémoire}
\textbf{LA SORTIE CHANGE AU COUP DE CLOCK}

Une mémoire change au \textbf{flanc actif d'horloge}. Dans une mémoire, on stocke une valeur et lorsque flan actif d'horloge, mise a jour de la sortie.

\subsubsection{Longue mémoire}
\textbf{CHANGE AU COUP DE CLOCK AVEC CONDITION FIXEE}

Dans la partie rising edge, on va mettre un If et si la condition choisie est respectée on affecte une nouvelle valeur.

\subsubsection{RTL}

[Combinatoire]==>[Mémoire]==>[Combinatoire]==>[Mémoire]






----------------------------------------------

----------------------------------------------

??????????????Page 56 à rajouter??????????????


----------------------------------------------

----------------------------------------------

\subsection{Exercices}

\insertImage{Img/27.JPG}{0.6}{Question}{1}
\insertImage{Img/28.JPG}{0.8}{réponse}{1}
\insertImage{Img/29.JPG}{0.6}{réponse}{1}




\newpage
\section{Chapter 2 Conception hiérarchique}
%SN1-CH7_NS_2018
\begin{cisco}[Système de plus en plus complexe]
1) Pour répondre à la complexité qui augmente,on applique une technique de travail universelle:

Diviser pour rêgner = réduire la complexité en plusieurs sous-problèmes

2) Un système numérique sera découpé en plusieurs composants qui interagissent/communiquent.

Forme de découpage la plus fréquente: HIERARCHIQUE
\end{cisco}

\subsection{Concept de découpage hiérarchique}
 \insertImage{Img/30.JPG}{0.8}{découpage hiérarchique}{1}
 
\subsection{Communication}
 \insertImage{Img/31.JPG}{0.8}{Type de communication}{1}
 
 % résumé complet jusquà la page 4 SN1-CH7_NS_2018
 
 \subsection{A COMPLETER}
 
 







\newpage
\section{Types, Opérateurs et Conversions}
%SN1-CH2_NS_2018



VHDL est un langage \textbf{typé} où il est obligatoire de spécifier le type des \textbf{objets} utilisés.

\insertImage{Img/32.JPG}{0.8}{Les sous-types}{1}

%finit jusquà la page 3

%théorie jusqu'a 6

\subsection{types scalaires entiers}


\insertImage{Img/33.JPG}{0.8}{Types scalaires entiers}{1}

%%rajouter page 8 et 9

%%théorie donnée jusqu'au 25

























\end{document}
